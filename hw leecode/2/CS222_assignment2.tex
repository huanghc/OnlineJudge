\documentclass[12pt,a4paper]{article}
\usepackage{ctex}
\usepackage{amsmath,amscd,amsbsy,amssymb,latexsym,url,bm,amsthm}
\usepackage{epsfig,graphicx,subfigure}
\usepackage{enumitem,balance,mathtools}
\usepackage{wrapfig}
\usepackage{mathrsfs, euscript}
\usepackage[usenames]{xcolor}
\usepackage{hyperref}
%\usepackage{algorithm}
%\usepackage{algorithmic}
%\usepackage[vlined,ruled,commentsnumbered,linesnumbered]{algorithm2e}
\usepackage[ruled,lined,boxed,linesnumbered]{algorithm2e}

\newtheorem{theorem}{Theorem}[section]
\newtheorem{lemma}[theorem]{Lemma}
\newtheorem{proposition}[theorem]{Proposition}
\newtheorem{corollary}[theorem]{Corollary}
\newtheorem{exercise}{Exercise}[section]
\newtheorem*{solution}{Solution}

\renewcommand{\thefootnote}{\fnsymbol{footnote}}

\newcommand{\postscript}[2]
 {\setlength{\epsfxsize}{#2\hsize}
  \centerline{\epsfbox{#1}}}

\renewcommand{\baselinestretch}{1.0}

\setlength{\oddsidemargin}{-0.365in}
\setlength{\evensidemargin}{-0.365in}
\setlength{\topmargin}{-0.3in}
\setlength{\headheight}{0in}
\setlength{\headsep}{0in}
\setlength{\textheight}{10.1in}
\setlength{\textwidth}{7in}
\makeatletter \renewenvironment{proof}[1][Proof] {\par\pushQED{\qed}\normalfont\topsep6\p@\@plus6\p@\relax\trivlist\item[\hskip\labelsep\bfseries#1\@addpunct{.}]\ignorespaces}{\popQED\endtrivlist\@endpefalse} \makeatother
\makeatletter
\renewenvironment{solution}[1][Solution] {\par\pushQED{\qed}\normalfont\topsep6\p@\@plus6\p@\relax\trivlist\item[\hskip\labelsep\bfseries#1\@addpunct{.}]\ignorespaces}{\popQED\endtrivlist\@endpefalse} \makeatother
\begin{document}
\noindent

%========================================================================
\noindent\framebox[\linewidth]{\shortstack[c]{
\Large{\textbf{CS222 Homework 2}}\vspace{1mm}\\
Exercises for Algorithm Design and Analysis by Li Jiang, 2016 Autumn Semester}}
~\\
\begin{enumerate}

\item There are two sorted arrays nums1 and nums2 of size m and n respectively.

Find the median of the two sorted arrays. The overall run time complexity should be O(log (m+n)).

Example 1:

nums1 = [1, 3]

nums2 = [2]

The median is 2.0

Example 2:

nums1 = [1, 2]

nums2 = [3, 4]

The median is (2 + 3)/2 = 2.5

Input:

int nums1[]; int m;

int nums2[]; int n;

Output:

double median.

%Your answer should be written here.

~\\
~\\


\item Find the contiguous subarray within an array (containing at least one number) which has the largest sum.

For example, given the array [-2,1,-3,4,-1,2,1,-5,4], the contiguous subarray [4,-1,2,1] has the largest sum = 6.

Input:

int A[]: the input array.

int N: length of A.

Output:

return the largest sum.

%Your answer should be written here.

~\\
~\\


\item Given a non-empty array containing only positive integers, find if the array can be partitioned into two subsets such that the sum of elements in both subsets is equal.

Note:

Each of the array element will not exceed 100.

The array size will not exceed 200.

Example 1:

Input: [1, 5, 11, 5]

Output: true

Explanation: The array can be partitioned as [1, 5, 5] and [11].

Example 2:

Input: [1, 2, 3, 5]

Output: false

Explanation: The array cannot be partitioned into equal sum subsets.

Input:

int A[]: the input array.

int N: length of A.

Output:

return true or false.

%Your answer should be written here.

~\\
~\\

\end{enumerate}
%========================================================================
\end{document}
